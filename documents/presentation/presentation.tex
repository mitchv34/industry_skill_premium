\documentclass[notes,11pt, aspectratio=169]{beamer}

\usepackage{pgfpages}
% These slides also contain speaker notes. You can print just the slides,
% just the notes, or both, depending on the setting below. Comment out the want
% you want.
\setbeameroption{hide notes} % Only slide
%\setbeameroption{show only notes} % Only notes
% \setbeameroption{show notes on second screen=right} % Both

\usepackage{helvet}
% \usepackage[default]{lato}
\usepackage{array}
\usepackage[longnamesfirst]{natbib}
% \bibliographystyle{plain}
% \usepackage{apalike}
\bibliographystyle{apalike}
% \usepackage[natbib, maxcitenames=3, mincitenames=11, style=apa]{biblatex}

\usepackage{tikz}
\newcommand*\circled[4]{\tikz[baseline=(char.base)]{
    \node[shape=circle, fill=#2, draw=#3, text=#4, inner sep=2pt] (char) {#1};}}
\usepackage{verbatim}
\setbeamertemplate{note page}{\pagecolor{yellow!5}\insertnote}
\usetikzlibrary{positioning}
\usetikzlibrary{snakes}
\usetikzlibrary{calc}
\usetikzlibrary{arrows}
\usetikzlibrary{decorations.markings}
\usetikzlibrary{shapes.misc}
\usetikzlibrary{matrix,shapes,arrows,fit,tikzmark}
\usepackage{amsmath}
\usepackage{mathpazo}
\usepackage{hyperref}
\usepackage{lipsum}
\usepackage{multimedia}
\usepackage{graphicx}
\usepackage{multirow}
\usepackage{graphicx}
\usepackage{dcolumn}
\usepackage{bbm}
\usepackage{adjustbox} % Shrink stuff
\usepackage{cancel}
\newcolumntype{d}[0]{D{.}{.}{5}}

\usepackage{changepage}
\usepackage{appendixnumberbeamer}
\newcommand{\beginbackup}{
   \newcounter{framenumbervorappendix}
   \setcounter{framenumbervorappendix}{\value{framenumber}}
   \setbeamertemplate{footline}
   {
     \leavevmode%
     \hline
     box{%
       \begin{beamercolorbox}[wd=\paperwidth,ht=2.25ex,dp=1ex,right]{footlinecolor}%
%         \insertframenumber  \hspace*{2ex} 
       \end{beamercolorbox}}%
     \vskip0pt%
   }
 }
\newcommand{\backupend}{
   \addtocounter{framenumbervorappendix}{-\value{framenumber}}
   \addtocounter{framenumber}{\value{framenumbervorappendix}} 
}

% Fancy fit image command with optional caption
\makeatletter
\newcommand{\fitimage}[2][\@nil]{
  \begin{figure}
    \begin{adjustbox}{width=0.9\textwidth, totalheight=\textheight-2\baselineskip-2\baselineskip,keepaspectratio}
      \includegraphics{#2}
    \end{adjustbox}
    \def\tmp{#1}%
   \ifx\tmp\@nnil
      \else
      \caption{#1}
    \fi
  \end{figure}
}
\makeatother

\usepackage{graphicx}
\usepackage[space]{grffile}
\usepackage{booktabs}

% These are my colors -- there are many like them, but these ones are mine.
\definecolor{blue}{RGB}{0,114,178}
\definecolor{red}{RGB}{213,94,0}
\definecolor{yellow}{RGB}{240,228,66}
\definecolor{green}{RGB}{0,158,115}

% % Enviroments
% \newtheorem{defin}{Definition.}
% \newtheorem{teo}{Theorem. }
% \newtheorem{lema}{Lemma. }
% \newtheorem{coro}{C
% \begin{frame}{Modeling Choice}orolary. }
% \newtheorem{prop}{Proposition. }
% \theoremstyle{definition}
% \newtheorem{examp}{Example. }
% % \numberwithin{problem}{subsection} 

\hypersetup{
  colorlinks=false,
  linkbordercolor = {white},
  linkcolor = {blue}
}


%% I use a beige off white for my background
\definecolor{MyBackground}{RGB}{255,253,218}

%% Uncomment this if you want to change the background color to something else
%\setbeamercolor{background canvas}{bg=MyBackground}

%% Change the bg color to adjust your transition slide background color!
\newenvironment{transitionframe}{
  \setbeamercolor{background canvas}{bg=yellow}
  \begin{frame}}{
    \end{frame}
}

\setbeamercolor{frametitle}{fg=blue}
\setbeamercolor{title}{fg=black}
\setbeamertemplate{footline}[frame number]
\setbeamertemplate{navigation symbols}{} 
\setbeamertemplate{itemize items}{-}
\setbeamercolor{itemize item}{fg=blue}
\setbeamercolor{itemize subitem}{fg=blue}
\setbeamercolor{enumerate item}{fg=blue}
\setbeamercolor{enumerate subitem}{fg=blue}
\setbeamercolor{button}{bg=MyBackground,fg=blue,}
\setbeamercolor{theotem}{fg=blue} 

% If you like road maps, rather than having clutter at the top, have a roadmap show up at the end of each section 
% (and after your introduction)
% Uncomment this is if you want the roadmap!
% \AtBeginSection[]
% {
%   \begin{frame}
%       \frametitle{Roadmap of Talk}
%       \tableofcontents[currentsection]
%   \end{frame}
% }


\setbeamercolor{section in toc}{fg=blue}
\setbeamercolor{subsection in toc}{fg=red}
\setbeamersize{text margin left=1em,text margin right=1em} 

\newenvironment{wideitemize}{\itemize\addtolength{\itemsep}{10pt}}{\enditemize}

\usepackage{environ}
\NewEnviron{videoframe}[1]{
  \begin{frame}
    \vspace{-8pt}
    \begin{columns}[onlytextwidth, T] % align columns
      \begin{column}{.58\textwidth}
        \begin{minipage}[t][\textheight][t]
          {\dimexpr\textwidth}
          \vspace{8pt}
          \hspace{4pt} {\Large \sc \textcolor{blue}{#1}}
          \vspace{8pt}
          
          \BODY
        \end{minipage}
      \end{column}%
      \hfill%
      \begin{column}{.42\textwidth}
        \colorbox{green!20}{\begin{minipage}[t][1.2\textheight][t]
            {\dimexpr\textwidth}
            Face goes here
          \end{minipage}}
      \end{column}%
    \end{columns}
  \end{frame}
}

\title[]{\textcolor{blue}{Industry Heterogeneity and Wage Inequality}}
\author[MVB]{}
\institute[UW-Madison]{Mitchell Valdes-Bobes}

\date{\today}


\begin{document}
%%% TIKZ STUFF
\tikzset{   
        every picture/.style={remember picture,baseline},
        every node/.style={anchor=base,align=center,outer sep=1.5pt},
        every path/.style={thick},
        }
\newcommand\marktopleft[1]{%
    \tikz[overlay,remember picture] 
        \node (marker-#1-a) at (-.3em,.3em) {};%
}
\newcommand\markbottomright[2]{%
    \tikz[overlay,remember picture] 
        \node (marker-#1-b) at (0em,0em) {};%
}
\tikzstyle{every picture}+=[remember picture] 
\tikzstyle{mybox} =[draw=black, very thick, rectangle, inner sep=10pt, inner ysep=20pt]
\tikzstyle{fancytitle} =[draw=black,fill=red, text=white]
%%%% END TIKZ STUFF

% % Title Slide
\begin{frame}
	\maketitle
\end{frame}
% % Outline Slide
% \begin{frame}
% 	\frametitle{Roadmap of Talk}
% 	\tableofcontents
% \end{frame}

% INTRO
% \begin{transitionframe}
%   \begin{center}
%     { \Huge \textcolor{black}{Introduction and Motivation}}
%   \end{center}
% \end{transitionframe}
% \section{Motivation}
\begin{frame}{What this talk is about?}
    \begin{wideitemize}
        \item I want to explore how the differences in industries' workforce composition impact the increasing wage disparities between workers.
    \end{wideitemize}
\end{frame}
% Motivation Slide
\begin{frame}{Motivation}
\begin{wideitemize}
    \item Wage Inequality has risen since the 1980s.
    \item The distribution of wages inside firms does not follow the same trend as the entire economy.
    \item \cite{song2019firming} Show that a substantial part of the rise in dispersion happened between firms instead of within firms.
    % \begin{wideitemize}
    %     \item \textbf{ Sorting effect:} high-wage workers are increasingly likely to work for more productive firms.
    %     \item \textbf{ Segregation effect:} high-paying workers to be working with each other more frequently.
    % \end{wideitemize}
    \item At the same time there has been an increase in occupational, educational, and ability segregation of employees.
\end{wideitemize}
\end{frame}
\begin{frame}{Motivation}
\begin{wideitemize}
    \item I will focus on industry level educatiuonal segregation.
    \item CITE HAILTWANGER HERE!!!
\end{wideitemize}
\end{frame}

\begin{frame}{INDUSTRY TRENDS}
\begin{center}
FIUGURE WITH THREE INDUSTRIES
    \center
    % \resizebox{\linewidth}{!}{
    %     \includegraphics{document/presentation_02082022/figures/Screenshot 2022-02-08 112045.png}}
\end{center}

\end{frame}\begin{frame}{Industry Level Trends \hyperlink{sp-model-decomposition}{\beamergotobutton{Back}}}\label{frame-industry-trends}
  \fitimage[]{../images/trend_correlation_slides.pdf}
\end{frame}
  
\begin{frame}{Increasing Industry Heterogeneity}
  \begin{center}
    Say Why I will KORV.
    \end{center}
\end{frame}

\begin{frame}{Increasing Industry Heterogeneity}
  \begin{center}
    Say Why I will KORV.
    \end{center}
\end{frame}

\begin{frame}{Model}
    \begin{wideitemize}
        \item I will use the model by~\cite{krusell2000capital} (henceforth \textbf{KORV}).
        \item Allows me to decompose the change of the wage premium paid to skilled workers in to two effects:
        \begin{itemize}
            \item The effect of the relative supply of skilled to unskilled labor.
            \item The capital-skill complementary effect.
        \end{itemize}
    \end{wideitemize}
\end{frame}

\begin{frame}{LITERATURE}
  My work is related to....
  \begin{wideitemize}
      \item I will use the model by~\cite{krusell2000capital} (henceforth \textbf{KORV}).
      \item Allows me to decompose the change of the wage premium paid to skilled workers in to two effects:
      \begin{itemize}
          \item The effect of the relative supply of skilled to unskilled labor.
          \item The capital-skill complementary effect.
      \end{itemize}
  \end{wideitemize}
\end{frame}


\section{Model}
\begin{frame}{KORV}
    \begin{wideitemize}
    \item Two types of capital
    \begin{itemize}
        \item $k_s$, structures.
        \begin{itemize}
            \item Buildings.
        \end{itemize}
        \item $k_e$, equipment, with relative price equal to $1/q$
        \begin{itemize}
            \item Machines, computers, intellectual property.
        \end{itemize}
    \end{itemize}
    \item Two types of labor
        \begin{itemize}
            \item $u$ low-skilled labor.
              \begin{itemize}
                \item $u= \psi^u h_u$ where $h_u$ is hours (observed) and $\psi^u$ is the quality of low-skilled labor (unobserved). 
            \end{itemize}
            \item $s$ high-skilled labor.
            \begin{itemize}
                \item $s = \psi^S h_s$ where $h_s$ is hours (observed) and $\psi^s$ is the quality of high-skilled labor (unobserved). 
            \end{itemize}
        \end{itemize}
\end{wideitemize}
\end{frame}

\begin{frame}{KORV}
    \begin{wideitemize}
    \item There are three final goods:
    \begin{itemize}
        \item Consumption $c$
        \item Structure investment $i_s$
        \item Equipment investment $i_e$.
    \end{itemize}
    \item Aggregate production:
    \begin{equation}\label{eq:production}
      c_t + i_{e_t} + i_{s_t} = Y_t = A_t G(k_{s_t}, k_{e_t}, u_t, s_t)
    \end{equation}
    \end{wideitemize}
\end{frame}

\begin{frame}{Production function}
    \begin{wideitemize}
    \item The production function is:
      \begin{equation}\label{eq:production_fun}
        G(k_{s_t}, k_{e_t}, u_t, s_t) = k_{s_t}^\alpha\left( \mu u_t^\sigma + (1-\mu)\left(\lambda k_{s_t}^\rho (1-\lambda)s_t^\rho\right)^\frac{\sigma}{\rho}\right)^\frac{1-\alpha}{\sigma}
      \end{equation}
    \item $\sigma_{H} = 1/(1-\rho)$ is the elasticity between equipment and high-skilled.
    \item $\sigma_{L} = 1/(1-\sigma)$ is the elasticity between low-skilled and equipment + high-skilled.
    \item Firms solve the following profit maximization problem 
    \begin{equation}\label{eq:profit_max}
      \max_{k_{s_t}, k_{e_t}, u_t, s_t} G(k_{s_t}, k_{e_t}, u_t, s_t) - r_{s_t}k_{s_t} - r_{e_t}k_{e_t} - w_{u_t} h_{u_t} - w_{s_t} h_{s_t}
    \end{equation}
  \end{wideitemize}
\end{frame}

\begin{frame}{Production Function}
    \begin{wideitemize}
        \item My objective is to use this model to test whether the evolution of the change in the wage premium for skilled labor in different industries can be explained using the capital-skill complementarity hypothesis.
        
        % % \only<2->{
        \item I can observe, $\textcolor{black}{w_{u}}, \textcolor{black}{w_{s_t}}, k_{s_t}, \textcolor{black}{k_{e_t}}, \textcolor{black}{h_{u_t}}, \textcolor{black}{h_{s_t}}$
        % % }
        % % \only<3->{
        % \item $H = \psi_t^H h$ and $L = \psi_t^L \ell$ where $\textcolor{black}{\psi_t^H}, \textcolor{black}{\psi_t^L}$, are assumed to be a random process $\psi_t^i = \psi_0^i + \varepsilon$. \only<4->{ With $\epsilon\sim\mathcal{N}(0, \textcolor{black}{\eta}) $}
        % % }
        % % \only<4->{
        % \item $\textcolor{black}{\sigma}, \textcolor{black}{\rho}$ and $\textcolor{black}{\eta}$,  I take from original paper (for now).
        % % }
        % % \only<5->{
        % \item Finally I estimate $\textcolor{black}{\lambda}, \textcolor{black}{\psi_0^H}$ and $\textcolor{black}{\psi_0^L}$, using \textbf{SPML} method.
        % % }
        
    \end{wideitemize}
\end{frame}

\subsection{Skill-Premium}
\begin{frame}{Skill Premium in the Model}
    % \only<1->{
    % Define skill premium ($\omega$) as the ratio of the wages of skilled to unskilled.
    
    \begin{wideitemize}
    \item Assuming competitive markets, workers are paid their marginal products per unit, of work:
    % }
    \begin{equation*}
      \omega_t = \frac{w_{s_t}}{w_{u_t}} = \frac{G_{h_s}(k_{s_t}, k_{e_t}, u_t, s_t) }{G_{h_u}(k_{s_t}, k_{e_t}, u_t, s_t) }
    \end{equation*}
    % \only<3->{
      \item We can obtain the following (log-linearized) expression for $\omega_t$:
      \begin{equation}\label{eq:skill_premium_log_linear}
        \ln \omega_{t} \simeq \lambda \frac{\sigma-\rho}{\rho}\left(\frac{k_{e_t}}{s_{t}}\right)^{\rho}+(1-\sigma) \ln \left(\frac{h_{u_t}}{h_{s_t}}\right)+\sigma \ln \left(\frac{\psi^s_t}{\psi^u_t}\right)
      \end{equation}
      \item Which in turn can be written in terms of growth rates ($g_x$):
      \begin{equation}\label{eq:skill_premium_growth_rates}
        \begin{aligned}
        g_{\omega t} \simeq &(1-\sigma)\left(g_{h_{u_t}}-g_{h_{s_t}}\right)+\sigma\left(g_{\psi^s_t}-g_{\psi^u_t}\right) \\
        &+(\sigma-\rho) \lambda\left(\frac{k_{e_t}}{s_{t}}\right)^{\rho}\left(g_{k_{e_t}}-g_{h_{s_t}}-g_{\psi^s_t}\right) 
        \end{aligned}
      \end{equation}
    \end{wideitemize}
\end{frame}

\begin{frame}{Skill Premnium Decomposition}\label{sp-model-decomposition}
  
  We have decomposed the skill premium into three parts:
    \begin{wideitemize}
      \vspace{0.5cm}
      \only<1>{
      \item $(1-\sigma)(g_{h_{u_t}}-g_{h_{s_t}})$ depends on the difference of the growth rates of skilled and unskilled and labor.
      \vspace{0.5cm}
    \begin{wideitemize}
      \item If both types of labor are substitutes i.e $\sigma_u < 0 \implies (1-\sigma) < 0$
      \item If skilled labor grows at a faster rate than unskilled labor, then the skill premium decreases. \hyperlink{frame-industry-trends}{\beamerbutton{Data}}
    \end{wideitemize}      
      }\only<2>{
      \item $\sigma\left(g_{\psi^s_t}-g_{\psi^u_t}\right)$ depends on the growth rate of the productivity of skilled and unskilled and labor. 
      \vspace{0.5cm}
      \begin{wideitemize}
        \item  I follow KORV in making the following stochastic assumptions about labor productivity:
              \begin{equation}\label{eq:stochastic_labor_productivity}
                \psi^i_t = \psi^i_0 + \epsilon \qquad \epsilon \sim N(0, \eta_\omega^2) \qquad i\in\{s,u\}
              \end{equation}
        \item On average $\sigma (g_{\psi^s_t}-g_{\psi^u_t} )$ is constant over time and does not affect the growth rate of the skill premium.
      \end{wideitemize}  
      }\only<3>{
      \item $(\sigma-\rho) \lambda\left(\frac{k_{e_t}}{s_{t}}\right)^{\rho}\left(g_{k_{e_t}}-(g_{h_{s_t}}+g_{\psi_{s_t}})\right)$. This component depends on two factors:
      \vspace{0.5cm}
      \begin{enumerate}
        \item The growth rate of equipment relative to the growth rates of skilled labor input.
        \vspace{0.5cm}
        \begin{wideitemize}
          \item Characterize the capital-skill complementarity hypothesis as $\sigma > \rho$.
          \item If equipment capital grows faster than skilled labor, the skill-premium will increase.
        \end{wideitemize}
        \vspace{0.5cm}
        \item  The ratio of capital equipment to skilled labor 
        \vspace{0.5cm}
        \begin{wideitemize}
          \item  The effect will get larger (smaller) over time if $\rho > 0\:$ ($\rho < 0$). 
        \end{wideitemize}
      \end{enumerate}   
      }
\end{wideitemize}
\end{frame}

\section{Estimation}
\begin{frame}
  \frametitle{Estimation}

  \begin{wideitemize}
    \item I follow the same methodology as KORV to estimate the model parameters.
    \only<1>{
    \item To simplify notation :
    \begin{align*}
      \psi_t &= \{\psi^u_t, \psi^s_t\} \\
      X_t &= \{ k_{s_t} , k_{e_t}, h_{s_t}, h_{u_t}\} \\
      \Phi &= \{\alpha, \sigma, \rho, \mu, \lambda, \psi^u_0, \psi^s_0, \eta_\omega \}
    \end{align*}
    \item Any $\{\mu, \lambda, \psi^u_0,\mu, \lambda, \psi^u_0, \psi^s_0\}$ act as scalling parameters thus, one can be fixed.
    \item There are $7$ parameters to be estimated.
    }\only<2>{
    \item The parameters are estimated using the following structural equations:
    \begin{align*}
      A_{t+1} G_{k_s}(X_{t+1}, \psi_{t+1} \mid \Phi ) &= q_t A_{t+1} G_{k_s}(X_{t+1}, \psi_{t+1} \mid \Phi ) + (1-\delta_e)\left(\frac{q_t}{q_{t+1}}\right) + \nu_t\\
      \frac{w_{s_t}h_{s_t} + w_{u_t}h_{u_t} }{Y_t} &= lsh(X_{t}, \psi_{t} \mid \Phi ) \\
      \frac{w_{s_t}h_{s_t}}{w_{u_t}h_{u_t}} &= wbr(X_{t}, \psi_{t} \mid \Phi )
    \end{align*}
    }\only<3>{
      \item The estimation method is a two-stage simulated pseudo-maximum likelihood estimation (SPMLE).
      \item In the first stage labor input is condeir potentially endogenous and is inteumented using: both capital series, lagged capital series, lagged prices and indicators of the bussiness cycle.
      \item In the second stage:
      \begin{wideitemize}
        \item Taking the variance $\eta_\omega$ as given, for each date $t$ generate $S$ realizations of the model.
        \item For eacgh date $t$ calculate the mean and variance of the realizations.
        \item Minimize the distance between the first momens of the model and the data, using the second moment as a weighting matrix.
      \end{wideitemize}
    }
  \end{wideitemize}

\end{frame}

\section{Results}
\begin{frame}
  \frametitle{Results}
  \begin{wideitemize}
    \item First, I will show the results of the replication of the modelm using updated data.
    \item Second, I will show the results of apply the model to each industry.
  \end{wideitemize}
\end{frame}

\subsection{KORV Replication}
\begin{frame}
  \frametitle{KORV Replication}
  \only<1>{
  \begin{table}[h]
    \begin{center}
      \begin{tabular}{rrrrr}
  \hline\hline
   & \textbf{KORV} & \textbf{Repl.} & \textbf{Ext.} & \textbf{Ind.} \\
   & \texttt{63-92} & \texttt{63-92} & \texttt{63-18} & \texttt{88-18} \\\hline
  $\alpha$ & 0.117 & 0.113 & 0.118 & 0.08 \\
  $\sigma$ & 0.401 & 0.464 & 0.503 & 0.313 \\
  $\rho$ & -0.495 & -0.56 & -0.343 & -0.154 \\
  $\eta_\omega$ & 0.043 & 0.043 & 0.083 & 0.043 \\\hline\hline
\end{tabular}

      \caption{\label{tab:estimation_korv} Parameter estimates KORV model.}
    \end{center}
    \end{table}
  }\only<2>{
  % \begin{table}[h]
  %  \begin{center}
  %  \begin{tabular}{rcccc}
  \hline\hline
   & \textbf{KORV} & \textbf{Repl.} & \textbf{Ext.} & \textbf{Ind.} \\
   & \texttt{63-92} & \texttt{63-92} & \texttt{63-18} & \texttt{88-18} \\\hline
  $\sigma_s$ & 0.67 & 0.64 & 0.74 & 0.86 \\
  $\sigma_u$ & 1.67 & 1.86 & 2.01 & 1.45 \\\hline\hline
\end{tabular}

  %  \caption{\label{tab:estimation_elasticities_korv} Implied Elastities of Substitution}
  %  \end{center}
  %  \end{table}

  \begin{figure}[H]
    \centering
    \includegraphics[width=0.3\textwidth]{../images/fig:korv_estimation_ls_slides.pdf}
    \hfill
    \includegraphics[width=0.3\textwidth]{../images/fig:korv_estimation_wbr_slides.pdf}
    \hfill
    \includegraphics[width=0.3\textwidth]{../images/fig:korv_estimation_sp_slides.pdf}
    \caption{\label{fig:korv_estimation} The model Fit for the $1963$ - $1992$ period with KORV Data.}
  \end{figure}
  }\only<3>{
  \begin{figure}[H]
    \centering
    \includegraphics[width=0.3\textwidth]{../images/fig:updated_estimation_ls_slides.pdf}
    \hfill
    \includegraphics[width=0.3\textwidth]{../images/fig:updated_estimation_wbr_slides.pdf}
    \hfill
    \includegraphics[width=0.3\textwidth]{../images/fig:updated_estimation_sp_slides.pdf}
    \caption{\label{fig:korv_estimation_extended} The model Fit for the $1963$ - $2018$ period with Updated Data.}
  \end{figure}
  }\only<4>{
  \begin{figure}[H]
    \centering
    \includegraphics[width=0.3\textwidth]{../images/fig:updated_ind_estimation_ls_slides.pdf}
    \hfill
    \includegraphics[width=0.3\textwidth]{../images/fig:updated_ind_estimation_wbr_slides.pdf}
    \hfill
    \includegraphics[width=0.3\textwidth]{../images/fig:updated_ind_estimation_sp_slides.pdf}
    \caption{\label{fig:korv_estimation_extended_industry} The model Fit for the $1988$ - $2018$ period with Updated Data.}
  \end{figure}
  }
\end{frame}


\subsection{Industry Level Results}
\begin{frame}
  \frametitle{Industry Level Results}
  

\end{frame}

% \section{Data}
% \begin{frame}[label=data]{Data}
% \centering
%     \begin{wideitemize}
%         \item A crucial step to take this model to data is to extend the original KORV data to the present.
%         \item I follow the procedure by \cite{ohanian2021revisiting}.
%     \item Capital stock series (both equipment and structures) are constructed from investment series from NIPA.
%     \hyperlink{capital_data}{\beamergotobutton{Capital Data}}
%     \item Labor data (wages and labor input of skilled and unskilled workers) is constructed from the ASEC of the CPS \hyperlink{labor_data}{\beamergotobutton{Labor Data}}
%     \item Both capital and labor series can be also constructed at the industry level, which will allow to estimate the model for each industry (2 or 3 NAICS digits).
%     \end{wideitemize}
% \end{frame}

% \begin{transitionframe}
%   \begin{center}
%     { \Huge \textcolor{black}{Simulations}}
%   \end{center}
% \end{transitionframe}

% \section{Simulation}


% \begin{frame}[label=sim_result]{Simulation Results}
%     \begin{wideitemize}
        
%     % \only<1-4>{
%         \item \hyperlink{skil_prem<2>}{\beamergotobutton{Relative Quantity Effect}} $\sigma = 0.436 \quad \implies \quad \sigma^L > 1$
%         % \only<2-4>{
%         % \begin{center}
%         % \includegraphics[width=.7\textwidth]{document/figures/quantity_effect.pdf}
%         % \end{center}
%         % \item \hyperlink{skil_prem<4>}{\beamergotobutton{Capital-Skill Complementarity Effect}} $\rho =  -0.517 \quad \implies \quad \sigma > \rho \quad \implies \quad \sigma_L > \sigma_H$
%         % % \only<4>{
%         % \begin{center}
%         % \includegraphics[width=.7\textwidth]{document/figures/cap_skill_complementarity.pdf}
%         % \end{center}
%     %     }
%     % }
    
%     \end{wideitemize}
% \end{frame}

% \begin{frame}[sim_result]{Simulation Results\hyperlink{sim_korv_data}{\beamergotobutton{KORV Data}}}
%     % \begin{center}
%     %     % \caption{Simulated Skill Premium from the Model}
%     %     \includegraphics[width=.65\textwidth]{document/figures/sim_skill_premium.pdf}
%     % \end{center}
% \end{frame}


% \section{Future Work}
% \begin{frame}{Future Work}
%     \begin{wideitemize}
%         \item Estimate the model at the industry level. 
%         \item Analyze how much variation in skill premium at the industry level is due to differences in estimated parameters.
%         \item Try to implement a mechanism that links labor force composition, skill premium, and skill-biased technological from \textbf{KORV}.
%     \end{wideitemize}
% \end{frame}

\section*{References}
\begin{frame}{References}
    \bibliographystyle{chicago}
    % \bibliography{../manuscript/references.bib}
\end{frame}


% \appendix
% \subsection{Capital}
% \begin{frame}[label=capital_data]{Appendix: Capital Data \hyperlink{data}{\beamergotobutton{Back}}}
%     \begin{wideitemize}
%         \item To obtain annual series for both types of capital I use the Perpetual Inventory Method (PIM):
%         $$k_{t+1}^{i} = (1 - \delta_{t}_{i})k_{t}^{i} + x_{t}^{i} \qquad i \in \{e,s\}$$
%         \item Capital investment data for both equipment and structures dating from $1947$ are from the National Income and Product Accounts (NIPA).
%         \item To iteratively generate the structures capital stock we need:
%         \begin{itemize}
%             \item The depreciation rates $\delta$ obtained, an initial value for $K_{1947}^i$, a deflator.
%         \end{itemize}
%     \end{wideitemize}
% \end{frame}

% \begin{frame}{Appendix: Capital Data}
%     \begin{figure}[h]
%     \caption{Updated Capital Structures and Equipment series (normalized to 1963=1)}
%     \centering
%     \includegraphics[width=.85\textwidth]{document/figures/capital_data_1.pdf}
%     \end{figure}
% \end{frame}

% \begin{frame}{Appendix: Capital Data}\hyperlink{data}{\beamergotobutton{Back}}
%     \begin{figure}[h]
%     \caption{Ratios between purchases and prices (normalized to 1963=1)}
%     \centering
%     \includegraphics[width=.85\textwidth]{document/figures/capital_data_2.pdf}
%     \end{figure}
% \end{frame}

% \subsection{Labor Input and Wages}

% \begin{frame}[label=labor_data1]{Appendix: Labor Input and Wages}
%     \begin{wideitemize}
%       \item  To construct the wage and labor input series, I use the march supplement of the current population survey (CPS). \hyperlink{labor_data_appendix1}{\beamergotobutton{Sub Sample}}
%     \item  For each person, record: \hyperlink{labor_data_appendix2}{\beamergotobutton{Variables}}
%     \begin{itemize} 
%     \item their personal characteristics, employment statistics, income, CPS personal supplement weights.
%     \end{itemize}
%     \item Workers are assigned to one of $264$ groups based on their characteristics. \hyperlink{labor_data_appendix3}{\beamergotobutton{Groups}}
%     \item I calculated the average hours worked each year and the hourly wage for each group. \hyperlink{labor_data_appendix4}{\beamergotobutton{Group Averages}}
%     \item Finally, I aggregate annual data on wages and hours distinguishing groups with college degree ($H$) and non-college degree ($L$) \hyperlink{labor_data_appendix5}{\beamergotobutton{Aggregation}}

%     \end{wideitemize}
% \end{frame}
% \begin{frame}{Appendix: Labor Data}
%     \begin{figure}[h]
%     \caption{Wages and Labor Input (normalized to 1963=1)}
%     \centering
%     \includegraphics[width=1\textwidth]{document/figures/labor_data.pdf}
%     \end{figure}
% \end{frame}


% \begin{frame}[label=labor_data_appendix1]{Appendix: CPS Sub-sample \hyperlink{labor_data1}{\beamergotobutton{Back}}}
%   I include all observations excluding agents:  
%   \begin{itemize}
%         \item younger than $16$ or older than $70$.
%         \item unpaid family workers.
%         \item those working in the military.
%         \item those who report working less than $40$ weeks a year and/or $35$ hours a week.
%         \item individuals with allocated income.
%         \item those with hourly wages below half of the minimum federal wage rate.
%         \item those whose weekly pay was less than $\$62$ in $1980$ dollars (equivalent to $\$(2.022 \times  62)$ in $1999$ dollars).
%         \item those did not report their education level
%         \item self-employed are excluded for wage sample but used when constructing labor input series.
%   \end{itemize}
% \end{frame}
% \begin{frame}[label=labor_data_appendix2]{Appendix: Variables\hyperlink{labor_data1}{\beamergotobutton{Back}}}
    
%     \begin{wideitemize}
%     \item their personal characteristics:
%     \begin{itemize}
%         \item age, sex, race 
%     \end{itemize}
%     \item employment statistics: 
%     \begin{itemize}
%         \item employment status (\texttt{empstat}).
%         \item class of worker (\texttt{classwly}). 
%         \item weeks worked last year (\texttt{wkswork1} and \texttt{wkswork2}).
%         \item usual hours worked per week last year (\texttt{uhrsworkly} and hours work last week \texttt{ahrsworkt}. 
%     \end{itemize}
%     \item income:
%         \begin{itemize}
%             \item total wage and salary income \texttt{incwage}.
%         \end{itemize}
%     \item CPS personal supplement weights: \texttt{asecwt}.
%     \end{wideitemize}
% \end{frame}
% \begin{frame}[label=labor_data_appendix3]{Appendix: Groups \hyperlink{labor_data1}{\beamergotobutton{Back}}}
  
%     To homogenize the data I create the following groups based on individual characteristics:
%     \begin{itemize}
%     \item Age is divided into 11 five-year groups:
%     \begin{itemize}
%         \item $16-20 \ldots 66-70$. 
%     \end{itemize}
%     \item Race is divided into three: 
%     \begin{itemize}  
%         \item {white}, {black}, {others}
%     \end{itemize}
%     \item  sex is divided into:
%     \begin{itemize}
%         \item male and female
%     \end{itemize}
%     \item  education is divided into four groups:
%     \begin{itemize}
%         \item {below high school}, {high school}, {
%         some college} and {college graduates and beyond}
%     \end{itemize}
%     \end{itemize}
% \end{frame}

% \begin{frame}[label=labor_data_appendix4]{Appendix: Group Averages\hyperlink{labor_data1}{\beamergotobutton{Back}}}
% \begin{wideitemize}
%     \item For every individual I create the following variables:
%     \begin{itemize}
%         \item $\ell_{i,t}$ the hours worked by individual $i$ in year $t$, is the product of hours worked per week times weeks worked that year.
%         \item $w_{i,t}$ the hourly wage of individual $i$ in year $t$, obtained by dividing yearly wage income by hours worked in year $t$.
%     \end{itemize}
%   \item Let $\mathcal{G}$ be the collection of all groups \item $\mu_{g,t} = \sum_{i\in g} \mu_{i,t}$ is the sum of CPS weight of the group.
%     \item Average hours worked for each group $g\in\mathcal{G}$: 
%     $$\ell_{g, t-1} = \frac{\sum_{i\in g}\ell_{i,t-1} \mu_{i,t}}{\mu_{g, t}} \quad \text{and} \quad
%     w_{g, t-1} = \frac{\sum_{i\in g}w_{i,t-1} \mu_{i,t}}{\mu_{g, t}}$$
% \end{wideitemize}
% \end{frame}

% \begin{frame}[label=labor_data_appendix5]{Appendix: Aggregation \hyperlink{labor_data1}{\beamergotobutton{Back}}}
%   \item  Partition the set $\mathcal{G}$ in two subsets $(\mathcal{H}, \mathcal{L})$ based on education (college graduates and non-college graduates). 
       
%       \item Total labor input:
%         $$L^H_{t-1} = \sum_{g \in\mathcal{H}} \ell_{g, t-1} \mu_{g,t} \quad \text{and} \quad L^L_{t-1} = \sum_{g \in\mathcal{L}} \ell_{g, t-1} \mu_{g,t}$$
        
%         \item and wages for each skill level:
        
%         $$W^H_{t-1} = \frac{\sum_{g \in \mathcal{H}} w_{g, t-1} \elll_{g, t-1} \mu_{g,t}}{L^H_{t-1}}\quad \text{and} \quad W^L_{t-1} = \frac{\sum_{g \in \mathcal{L}} w_{g, t-1} \ell_{g, t-1} \mu_{g,t}}{L^L_{t-1}}$$   
% %    
% \end{frame}

% \begin{frame}[label=sim_korv_data]{Simulation using KORV Data and Parameters\hyperlink{sim_result}{\beamergotobutton{Back}}}
%     \begin{center}
%         % \caption{Simulated Skill Premium from the Model}
%         \includegraphics[width=.65\textwidth]{document/figures/sim_skill_premium_korv.pdf}
%     \end{center}
% \end{frame}

\end{document}